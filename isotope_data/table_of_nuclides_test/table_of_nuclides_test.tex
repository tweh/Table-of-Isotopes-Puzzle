% !TeX program = xelatex
\documentclass[tikz, border=10mm]{standalone}

\usepackage[l3]{csvsimple}

\usetikzlibrary{calc,shadings}

\usepackage[sfdefault,lining]{FiraSans}

\usepackage[version=4]{mhchem}

\newlength{\tilesize}
\setlength{\tilesize}{14mm}

% colors of the source
\colorlet{stablecolor}{black!90}
\colorlet{alphacolor}{yellow!30}
\colorlet{betaminuscolor}{blue!15}
\colorlet{betapluscolor}{orange!35}

\tikzset{
  x = \tilesize,
  y = \tilesize,
  every picture/.style = {
    line width = 1pt
  },
  tile node/.style={
    rectangle,
    draw,
    minimum size = \tilesize,
    inner sep = 0pt,
    align = center,
  },
  o/.style = {
    fill = betapluscolor,
  },
  g/.style = {
    fill = alphacolor,
  },
  b/.style = {
    fill = betaminuscolor,
  },
  g+b/.style = {
    fill = alphacolor,
    fill half = betaminuscolor,
  },
  g+o/.style = {
    fill = alphacolor,
    fill half = betapluscolor,
  },
  b+o/.style = {
    fill = betaminuscolor,
    fill half = betapluscolor,
  },
  s/.style = {
    fill = stablecolor,
    text = white,
  },
  fill half/.style = {
    path picture = {% thanks to https://tex.stackexchange.com/a/584557/4918
      \fill [#1] (path picture bounding box.south west)
        -- (path picture bounding box.south east)
        -- (path picture bounding box.north east)
        -- cycle;
    }
  },
}

\begin{document}
\begin{tikzpicture}
  % axis
  \draw [->] (119,76) -- (149,76) node [below] {\textbf{N}};
  \foreach \n in {120,...,148}
    \draw (\n,76) -- ++(0,-0.2) node [below] {\n};
  \draw [->] (119,76) -- (119,96) node [left] {\textbf{Z}};
  \foreach \z in {77,...,95}
    \draw (119,\z) -- ++(-0.2,0) node [left] {\z};
  % tiles
  \csvreader[
    separator = semicolon,
    head to column names,
    head to column names prefix = DATA,
  ]
  {../isotope_data.csv}{
    decay text=\DATAdecays
  }{
    \node at (\DATAN,\DATAZ) [tile node, \DATAcolor] {%
%      \textbf{\textsuperscript{\DATAA}\DATAsymbol}\\
      \textbf{\ce{^{\DATAA}_{\DATAZ}\DATAsymbol}}\\[0.5ex]
%      \DATAN,\DATAZ\\
      \footnotesize\DATAdecays
    };
  }
\end{tikzpicture}
\end{document}